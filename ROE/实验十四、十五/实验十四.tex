%---------------------导言区---------------------------%
\documentclass[12pt,a4paper,UTF8]{ctexart}
	%10pt:正文字体为12pt,缺省为10pt;各层级字体大小会根据正文字体自动调整
	%a4paper:纸张大小a4;
	%UTF8:中文要求
%\usepackage{syntonly}
%\syntaxonly%加快编译速度
\usepackage{geometry}%用于设置上下左右页边距
	\geometry{left=2.5cm,right=2.5cm,top=3.2cm,bottom=2.8cm}
\usepackage{multirow}
\usepackage{xeCJK,amsmath,paralist,enumerate,booktabs,multirow,graphicx,float,subfig,setspace,listings,lastpage,hyperref,gensymb}
	%xeCJK:中文字体(如楷体,作者和机构需要用到)的设置
	%amsmath:数学公式
	%paralist,enumerate:自定义项目符号
	%booktabs:三线图,论文常用的表格风格
	%multirow:复杂表格
	%graphicx,float: 插入图片
	%subfig:并排排版图片以及强制图表显示在“这里”[H]
	%setspace:设置行间距等功能
	\setlength{\parindent}{2em}%正文首行缩进两个汉字
	%listings:用于排版各种代码;比如matlab的代码
	%\lstset{language=Matlab}%matlab代码
	%lastpage:获取总页数;
	%hyperref:超链接,和lastpage搭配.
\usepackage{fancyhdr}
	%fancyhdr:一个很强大的宏包,用于自定义设计页面风格并命名以供调用。
	\pagestyle{fancy}
	\rhead{实验十四~直流电桥测量电阻}
	\lhead{普通物理实验\uppercase\expandafter{\romannumeral1}实验报告}
	\cfoot{\thepage}  
		%分别是右页眉、左页眉、右页脚
	\renewcommand{\headrulewidth}{0.4pt}
	\renewcommand{\theenumi}{(\arabic{enumi})}

\setCJKmainfont{FZSSK.TTF}[ItalicFont=FZKTK.TTF, BoldFont=FZHTK.TTF]
%中文字体设置:使用开源字体方正书宋,方正楷体和方正黑体



%%%%%%%%%%%%%%%%%%%%%%%%%%%%%%%%%%%%%%%%%%%%%%%%%%%%%%%%%%
%%%%%%%%%%%%%%%%%%%%%%%%%正文开始%%%%%%%%%%%%%%%%%%%%%%%%%%
%%%%%%%%%%%%%%%%%%%%%%%%%%%%%%%%%%%%%%%%%%%%%%%%%%%%%%%%%%

\begin{document}

%%begin-------------------标题与信息-----------------------%%

%%标题
\begin{center}
\LARGE\textbf{实验十四~直流电桥测量电阻}
\end{center}

%%信息
\begin{doublespacing}
	%doublespacing:手动两倍行距
	\centering
	\begin{tabular}{ll}
	 & \\
	{\CJKfontspec{STKAITI.TTF} 实验人:钟易轩}  & {\CJKfontspec{STKAITI.TTF}指导教师:张晓东}\\
	{\CJKfontspec{STKAITI.TTF} 组号:九组七号} & {\CJKfontspec{STKAITI.TTF}学号:2000012706}\\
	{\CJKfontspec{STKAITI.TTF} 实验时间:2021年11月26日} &{\CJKfontspec{STKAITI.TTF} 实验地点:物理楼南楼~233}
	\end{tabular}
\end{doublespacing}

%%end-------------------标题与信息-----------------------%%
\subsection*{【实验目的】}
\begin{enumerate}[(1)]
\item 学习直流电桥的基本原理;
\item 误差分析.
\end{enumerate}
\subsection*{【仪器用具】}
$ZX96$型电阻器3个,直流指针式检流计,待测电阻3个,直流电源,开关,导线.
\subsection*{【数据处理】}
\subsubsection*{1.测$R_x$及电桥灵敏度$S$}
首先用万用表测量直流电源的电压,调节到$E=3.98V$,再观察检流计的各项参数,得到分度值为$1.3\times10^{-6}$(安/格),内阻$R_g=44\Omega$.则有检流计的灵敏度为$S_i=\dfrac{1}{1.3\times10^{-6}}\approx7.7\times10^{5}$(格/安).测量数据如表1所示.
\begin{table}[htbp]
\centering
\caption{$R_x$与$S$测量表}
\scalebox{1.1}{
\begin{tabular}{|c|c|c|c|c|c|c|c|c|}
\hline
$R_{xi}$ & $R_1$/$R_2$ &$R_0$/$\Omega$&$R_0^{\prime}$/$\Omega$ &  $\Delta n$/格 &    $R_x$/$\Omega$  &$\Delta R_x$/$\Omega$ &  $S$/格 &$\sigma_{R_{xi}}/\Omega$ \\
\hline
\multirow{2}{*}{$R_{x1}$}&\multirow{2}{*}{$500\Omega$/$500\Omega$}&\multirow{2}{*}{47.18}&47.2&左0.7&\multirow{2}{*}{47.18}&\multirow{2}{*}{0.1}&\multirow{2}{*}{$1.7\times10^{3}$}&\multirow{2}{*}{0.06} \\
\cline{4-5}
& & &47.1&右3.0& & & &\\
\hline
\multirow{3}{*}{$R_{x2}$}&$50\Omega$/$500\Omega$&2991.5&2971.5&5.0&299.15&20.0&$7.5\times10^{2}$&0.31 \\
\cline{2-9}
&$500\Omega$/$500\Omega$&298.8&298.0&4.0&298.8&0.8&$1.5\times10^{3}$&\multirow{2}{*}{0.16} \\
\cline{2-8}
&$500\Omega$/$500\Omega$&298.8&298.0&4.2&298.8&0.8&$1.6\times10^{3}$& \\
\hline
$R_{x3}$&$500\Omega$/$500\Omega$&4210.0&4270.0&4.0&4210.0&60.0&$2.8\times10^{2}$&6.2 \\
\hline
\end{tabular}  }
\end{table}
\par
由于平衡电桥测量电阻的误差有两大来源,一是桥臂电阻带来的误差,二是电桥灵敏度带来的误差.因此经过合成,得到$R_x$的不确定度表达式如下
\begin{equation}
\sigma_{R_x}=\sqrt{(\delta R_x)^2+(\frac{R_0}{R_2})^2\sigma_{R_1}^2+(\frac{R_0R_1}{R_2^2})^2\sigma_{R_2}^2+(\frac{R_1}{R_2})^2\sigma_{R_0}^2} \tag{14.1}
\end{equation}
\par
其中$\delta R_x=\dfrac{0.2\Delta R_x}{\Delta n}=\dfrac{0.2R_1\cdot\Delta R_0}{\Delta n\cdot R_2}$.\par
由于$R_1$、$R_2$与$R_0$都是由$ZX96$型电阻器测量一次得到的值,因此其不确定度主要由电阻器的允差决定.表2是$ZX96$型电阻器各量程的允差.
\begin{table}[htbp]
\centering
\caption{$ZX96$型直流电阻器允差}
\begin{tabular}{|c|c|c|c|c|c|c|}
\hline
\textbf{挡位($\Omega$)}&$\times10k\Omega$&$\times 1k\Omega$&$\times100\Omega$&$\times10\Omega$&$\times1\Omega$&$\times0.1\Omega$ \\
\hline
\textbf{允差$e$}&$\pm0.1\%$&$\pm0.1\%$&$\pm0.1\%$&$\pm0.1\%$&$\pm0.5\%$&$\pm2\%$ \\
\hline
\end{tabular}
\end{table}
\par
利用表2数据与公式(14.1)计算$R_{x1}$的不确定度.
\begin{align*}
\sigma_{R_1}&=\frac{1}{\sqrt{3}}(500\times0.1\%)=0.29(\Omega) \\
\sigma_{R_2}&=\frac{1}{\sqrt{3}}(500\times0.1\%)=0.29(\Omega) \\
\sigma_{R_0}&=\frac{1}{\sqrt{3}}(40\times0.1\%+7\times0.5\%+0.1\times2\%)=0.045(\Omega) \\
\delta R_x&=\frac{0.2\times500\times0.08}{3.0\times500}=5.3\times10^{-3} \\
\sigma_{R_{x1}}&=\sqrt{(5.3\times10^{-3})^2+(\frac{47.18}{500})^2\times0.29^2+(\frac{47.18\times500}{500^2})^2\times0.29^2+(0.045)^2}=0.06(\Omega)
\end{align*}
\par
之后的$R_{x2}$和$R_{x3}$可用同样的步骤去计算.但是在$R_{x2}$的测量数据中,有两组数据是将$R_1$和$R_2$的位置调换之后进行测量的,这里我们可以根据这两组数据用以下公式计算$R_{x2}$的不确定度.
\begin{equation}
\sigma_{R_{x2}}=\sqrt{k\cdot\delta R_x^2+\frac{1}{4}\frac{R_{01}}{R_{02}}\sigma_{R_{02}}^2+\frac{1}{4}\frac{R_{02}}{R_{01}}\sigma_{R_{01}}^2} \tag{14.2}
\end{equation}
\par
则得出交换桥臂法测出的$R_{x2}=(298.8\pm0.2)\Omega$.
\subsubsection*{2.不同参量下的电桥灵敏度}
改变测量条件,观察其对电桥灵敏度的影响,可得出表3(未知电阻用$R_{x2}$).
\begin{table}[htbp]
\centering
\caption{不同参量下的电桥灵敏度}
\begin{tabular}{|c|c|c|c|c|c|c|}
\hline
测量条件&$R_0/\Omega$&$R_0^{\prime}/\Omega$&$\Delta n$/格&$R_x/\Omega$&$\Delta R_x/\Omega$&$S$/格 \\
\hline
$E=4.01V,\frac{R_1}{R_2}=\frac{500\Omega}{500\Omega},R_h=0\Omega$&298.8&299.8&5.0&298.8&1.0&$1.5\times10^{3}$ \\
\hline
$E=2.02V,\frac{R_1}{R_2}=\frac{500\Omega}{500\Omega},R_h=0\Omega$&298.8&300.8&5.0&298.8&2.0&$7.5\times10^2$ \\
\hline
$E=4.01V,\frac{R_1}{R_2}=\frac{500\Omega}{5000\Omega},R_h=0\Omega$&2988.0&2938.0&5.5&298.8&50.0&$3.3\times10^2$ \\
\hline
$E=4.01V,\frac{R_1}{R_2}=\frac{500\Omega}{500\Omega},R_h=2.988k\Omega$&298.8&292.8&4.0&298.8&6.0&$2.0\times10^2$\\
\hline
\end{tabular}
\end{table}
\par
上表中的数据反映了电桥灵敏度的大小与电源电压等有关,因此正好符合了$S$的决定式.
\begin{equation}
S=\frac{S_i\cdot E}{R_1+R_2+R_0+R_x+(R_g+R_h)(2+\frac{R_1}{R_x}+\frac{R_0}{R_2})} \tag{14.3}
\end{equation}
\subsection*{【思考题】}
下列因素是否会加大测量误差?
\begin{enumerate}[(1)]
\item 电源电压大幅度下降;\\
答:会加大误差.由公式14.3得,当电源电压大幅度下降时,电桥灵敏度也会大幅度下降,又由于$S$与$\delta R_x$呈负相关,因此由公式14.1得,$\sigma_{R_x}$增大.
\item 电源电压稍有波动;\\
答:由于只是微小波动,因此误差也会在一定范围内进行波动,但是波动幅度小便可以忽略.
\item 在测量较低电阻时,导线电阻不可忽略;\\
答:由于导线电阻不可忽略,则最后计算所得值中涵盖了导线电阻的值,因此会加大误差.
\item 检流计零点没有调准;\\
答:会加大误差.
\item 检流计灵敏度不够高.\\
答:当检流计灵敏度低时,由公式14.3得电桥灵敏度降低,由(1)的分析可知,$\sigma_{R_x}$会增加.
\end{enumerate}

\subsection*{【分析与讨论】}
1.在电阻不确定度中具有影响的是这些物理量:$\frac{R_1}{R_2},S,R_x,\frac{R_0}{R_2},\frac{R_0R_1}{R_2^2},\sigma_{R_1},\sigma_{R_2},\sigma_{R_0}$.\par
对于$\sigma_{R_1},\sigma_{R_2}$和$\sigma_{R_0}$来说,由于它们的大小与电阻箱的允差有关,因此可以选择较小允差的电阻箱,以此来减小这三个电阻的不确定度;对于各电阻的大小,尽量保持$\frac{R_1}{R_2}$的值为1;对于$S$的大小,可通过公式14.3的各参量进行调节,使$S$的值变大,比如选择灵敏度更高的检流计以及将电压变大.\par
2.对于表1中的数据,以$R_{x1}$为例,其测量所得的灵敏度$S=1.7\times10^3$,若将各参量代入公式14.3可得$S=1.85\times10^3>1.7\times10^3$,在计算了多组数据后发现,测量所得的灵敏度皆小于公式所得的灵敏度,原因应该是导线电阻与接触电阻使得公式14.3中的分母增大,导致其值变小.

\subsection*{【收获与感想】}
在使用直流电桥测量电阻时,一开始没把$R_h$调至零,导致检流计偏转极其微弱,以后在进行实验之前应该在脑袋里面将步骤想好,尽量做到细节完满;听完老师讲解误差分析之后,自己对误差分析的理解又加深了一点.






\end{document}