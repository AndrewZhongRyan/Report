%---------------------导言区---------------------------%
\documentclass[12pt,a4paper,UTF8]{ctexart}
	%10pt:正文字体为12pt,缺省为10pt;各层级字体大小会根据正文字体自动调整
	%a4paper:纸张大小a4;
	%UTF8:中文要求
%\usepackage{syntonly}
%\syntaxonly%加快编译速度
\usepackage{geometry}%用于设置上下左右页边距
	\geometry{left=2.5cm,right=2.5cm,top=3.2cm,bottom=2.8cm}
\usepackage{xeCJK,amsmath,paralist,enumerate,booktabs,multirow,graphicx,float,subfig,setspace,listings,lastpage,hyperref,gensymb}
	%xeCJK:中文字体(如楷体,作者和机构需要用到)的设置
	%amsmath:数学公式
	%paralist,enumerate:自定义项目符号
	%booktabs:三线图,论文常用的表格风格
	%multirow:复杂表格
	%graphicx,float: 插入图片
	%subfig:并排排版图片以及强制图表显示在“这里”[H]
	%setspace:设置行间距等功能
	\setlength{\parindent}{2em}%正文首行缩进两个汉字
	%listings:用于排版各种代码;比如matlab的代码
	%\lstset{language=Matlab}%matlab代码
	%lastpage:获取总页数;
	%hyperref:超链接,和lastpage搭配.
\usepackage{fancyhdr}
	%fancyhdr:一个很强大的宏包,用于自定义设计页面风格并命名以供调用。
	\pagestyle{fancy}
	\rhead{实验十九~分光计的调节和掠入射法测量折射率}
	\lhead{普通物理实验\uppercase\expandafter{\romannumeral1}实验报告}
	\cfoot{\thepage}  
		%分别是右页眉、左页眉、右页脚
	\renewcommand{\headrulewidth}{0.4pt}
	\renewcommand{\theenumi}{(\arabic{enumi})}

\setCJKmainfont{FZSSK.TTF}[ItalicFont=FZKTK.TTF, BoldFont=FZHTK.TTF]
%中文字体设置:使用开源字体方正书宋,方正楷体和方正黑体



%%%%%%%%%%%%%%%%%%%%%%%%%%%%%%%%%%%%%%%%%%%%%%%%%%%%%%%%%%
%%%%%%%%%%%%%%%%%%%%%%%%%正文开始%%%%%%%%%%%%%%%%%%%%%%%%%%
%%%%%%%%%%%%%%%%%%%%%%%%%%%%%%%%%%%%%%%%%%%%%%%%%%%%%%%%%%

\begin{document}

%%begin-------------------标题与信息-----------------------%%

%%标题
\begin{center}
\LARGE\textbf{实验十九~分光计的调节和掠入射法测量折射率}
\end{center}

%%信息
\begin{doublespacing}
	%doublespacing:手动两倍行距
	\centering
	\begin{tabular}{ll}
	 & \\
	{\CJKfontspec{STKAITI.TTF} 实验人:钟易轩}  & {\CJKfontspec{STKAITI.TTF}指导教师:马文君}\\
	{\CJKfontspec{STKAITI.TTF} 组号:九组七号} & {\CJKfontspec{STKAITI.TTF}学号:2000012706}\\
	{\CJKfontspec{STKAITI.TTF} 实验时间:2021年12月3日} &{\CJKfontspec{STKAITI.TTF} 实验地点:物理楼南楼~333}
	\end{tabular}
\end{doublespacing}

%%end-------------------标题与信息-----------------------%%

\subsection*{【实验目的】}
	\begin{enumerate}[(1)]
		\item 了解分光计的结构、作用和工作原理;
		\item 掌握分光计的调节要求、方法和使用规范;
		\item 用分光计测定三棱镜的顶角;
		\item 用掠入射法测定三棱镜的折射率;
		\item 用最小偏向角方法测定物质折射率.
	\end{enumerate}
	
\subsection*{【仪器用具】}
	分光计,玻璃三棱镜,钠灯,汞灯,平面镜,毛玻璃,放大镜等.
\subsection*{【数据处理】}
\subsubsection*{1.测定玻璃三棱镜顶角}
转动望远镜,先使望远镜光轴与棱镜AB面垂直,记录下此时左右游标的读数$\theta_1^{\prime}$,$\theta_1^{\prime \prime}$.然后转动望远镜,使其光轴与AC面垂直,记下两边游标读数$\theta_2^{\prime}$,$\theta_2^{\prime \prime}$,重复测量三次.数据如表1所示.
\begin{table}[htbp]
\centering
\caption{测定玻璃三棱镜顶角数据表}
\scalebox{1}{
\begin{tabular}{cccccc}
\toprule
i&$\theta_1^{\prime}$&$\theta_1^{\prime \prime}$&$\theta_2^{\prime}$&$\theta_2^{\prime \prime}$&$\psi_i$ \\
\hline
1&$333^{\circ}52^{\prime}$&$153^{\circ}45^{\prime}$&$213^{\circ}46^{\prime}$&$33^{\circ}50^{\prime}$&$120^{\circ}30^{\prime \prime}$ \\
\hline
2&$35^{\circ}8^{\prime}$&$215^{\circ}3^{\prime}$&$275^{\circ}4^{\prime}$&$95^{\circ}4^{\prime}$&$120^{\circ}1^{\prime}30^{\prime \prime}$ \\
\hline
3&$351^{\circ}26^{\prime}$&$171^{\circ}20^{\prime}$&$232^{\circ}22^{\prime}$&$51^{\circ}23^{\prime}$&$119^{\circ}30^{\prime}30^{\prime \prime}$ \\
\bottomrule
\end{tabular}}
\end{table}
\par
利用上述数据就可以求出A的大小.
\begin{align*}
\bar\psi&=\dfrac{1}{3}(120^{\circ}30^{\prime \prime}+120^{\circ}1^{\prime}30^{\prime \prime}+119^{\circ}30^{\prime}30^{\prime \prime})=119^{\circ}50^{\prime}50^{\prime \prime} \\
A&=180^{\circ}-119^{\circ}50^{\prime}50^{\prime \prime} =60^{\circ}9^{\prime}10^{\prime \prime} =60.15^{\circ} =1.05\mathrm{rad}
\end{align*}
\par
接下来再求A的不确定度$\sigma_A$.由于$\sigma_A=\sqrt{\sigma_{\bar A}^2+e^2/3}$,且有允差$e=0^{\circ}1^{\prime}$,$\sigma_{\bar A}=\sigma_{\bar \psi}$,则有
\begin{align*}
\sigma_{\bar \psi}&=\sqrt{\frac{\sum_{i=1}^3(\psi_i-\bar \psi)^2}{(3-1)\times3}} \\
&=0^{\circ}10^{\prime}10^{\prime \prime} \\
\sigma_A&=\sqrt{\sigma_{\bar\psi}^2+e^2/3} \\
&=0^{\circ}10^{\prime}11^{\prime \prime}
\end{align*}
\par
则最后$A=60^{\circ}9^{\prime}10^{\prime \prime}\pm0^{\circ}10^{\prime}11^{\prime \prime}$
\subsubsection*{2.用掠入射法测定三棱镜的折射率}
移动望远镜找到明暗分界线,用$PP^{\prime}$线对准明暗分界线,记下左右游标读数$\theta_3^{\prime}$、$\theta_3{\prime \prime}$.再将望远镜转动至AC面的法线位置,记下左右游标读数$\theta_4^{\prime}$、$\theta_4^{\prime \prime}$,重复测量三次.数据如表2所示.\par
其中折射率的计算有公式如下,
\begin{equation}
n=\sqrt{1+(\frac{\cos A+\sin \phi}{\sin A})^2} \tag{19.1}
\end{equation}
\begin{table}[htbp]
\centering
\caption{掠入法测定玻璃三棱镜折射率数据表}
\scalebox{1}{
\begin{tabular}{ccccccc}
\toprule
i&$\theta_3^{\prime}$&$\theta_3^{\prime \prime}$&$\theta_4^{\prime}$&$\theta_4^{\prime \prime}$&$\phi_i$&$n_i$ \\
\hline
1&$201^{\circ}57^{\prime}$&$22^{\circ}4^{\prime}$&$243^{\circ}20^{\prime}$&$63^{\circ}22^{\prime}$&$41^{\circ}20^{\prime}30^{\prime \prime}$&1.668 \\
\hline
2&$269^{\circ}24^{\prime}$&$89^{\circ}22^{\prime}$&$310^{\circ}49^{\prime}$&$130^{\circ}45^{\prime}$&$41^{\circ}24^{\prime}$&1.669 \\
\hline
3&$299^{\circ}31^{\prime}$&$119^{\circ}29^{\prime}$&$340^{\circ}58^{\prime}$&$160^{\circ}52^{\prime}$&$41^{\circ}25^{\prime}$&1.669 \\
\bottomrule
\end{tabular}}
\end{table}
\par
利用上述数据可以计算$\bar n$,得出$\bar n=1.669$.接下来计算折射率的不确定度.根据(19.1)式与方和根合成可得
\begin{equation}
\sigma_n=\sqrt{\left(\frac{(\cos A+\sin \phi)(1+\cos A\sin \phi)}{n\sin^3A}\sigma_A\right)^2+\left(\frac{\cos\phi(\cos A+\sin \phi)}{n\sin^2A}\sigma_{\phi}\right)^2} \tag{19.2}
\end{equation}
\par	
其中$\sigma_{\phi}$的计算方式与$\sigma_{\psi}$相同,得出$\sigma_{\phi}=0^{\circ}1^{\prime}29^{\prime \prime}$.再将这些数据代入(19.2)式中得到$\sigma_n=0.004$,则$n=(1.669\pm0.004)$.
\subsubsection*{3.用最小偏向角法测定三棱镜折射率}
由于时间有限,因此我只测了三条谱线的最小偏向角,分别是绿光(546.07nm)、钠黄光(579.07nm)和紫光(435.84nm).\par
其中计算折射率的公式为
\begin{equation}
n=\frac{\sin\frac{A+\delta}{2}}{\sin\frac{A}{2}} \tag{19.3}
\end{equation}
\par
由式(19.3)以及方和根合成公式可得$\sigma_n$的表达式为
\begin{equation}
\sigma_n=\sqrt{\left(\frac{\sin\frac{\delta}{2}}{2\sin^2\frac{A}{2}}\sigma_A\right)^2+\left(\frac{\cos\frac{A+\delta}{2}}{2\sin\frac{A}{2}}\sigma_{\delta}\right)^2} \tag{19.4}
\end{equation}
\par
\textcircled{1}先测绿光,数据如表3所示.
%%%%%
\begin{table}[htbp]
\centering
\caption{最小偏向角法测折射率数据表——绿光}
\scalebox{1}{
\begin{tabular}{ccccccc}
\toprule
i&$\theta_5^{\prime}$&$\theta_5^{\prime \prime}$&$\theta_6^{\prime}$&$\theta_6^{\prime \prime}$&$\delta_i$&$n_i$ \\
\hline
1&$282^{\circ}37^{\prime}$&$102^{\circ}35^{\prime}$&$336^{\circ}47^{\prime}$&$156^{\circ}41^{\prime}$&$54^{\circ}8^{\prime}$&1.676 \\
\hline
2&$326^{\circ}34^{\prime}$&$146^{\circ}30^{\prime}$&$20^{\circ}38^{\prime}$&$200^{\circ}32^{\prime}$&$54^{\circ}3^{\prime}$&1.675 \\
\hline
3&$7^{\circ}55^{\prime}$&$187^{\circ}48^{\prime}$&$61^{\circ}58^{\prime}$&$241^{\circ}54^{\prime}$&$54^{\circ}4^{\prime}30^{\prime \prime}$&1.676 \\
\bottomrule
\end{tabular}}
\end{table}
%%%%%
\par
其中$\bar\delta=54^{\circ}5^{\prime}10^{\prime \prime}$,$\bar n=1.676$,$\sigma_{\delta}=0^{\circ}1^{\prime}35^{\prime \prime}$,再将数据代入式(19.4)中,得到$\sigma_n=0.003$.因此$n=(1.676\pm0.003)$.\par
\textcircled{2}再测钠黄光,数据如表4所示.
%%%%%
\begin{table}[htbp]
\centering
\caption{最小偏向角法测折射率数据表——黄光}
\scalebox{1}{
\begin{tabular}{ccccccc}
\toprule
i&$\theta_7^{\prime}$&$\theta_7^{\prime \prime}$&$\theta_8^{\prime}$&$\theta_8^{\prime \prime}$&$\delta_i$&$n_i$ \\
\hline
1&$4^{\circ}55^{\prime}$&$184^{\circ}49^{\prime}$&$58^{\circ}36^{\prime}$&$238^{\circ}32^{\prime}$&$53^{\circ}42^{\prime}$&1.672 \\
\hline
2&$51^{\circ}59^{\prime}$&$231^{\circ}56^{\prime}$&$105^{\circ}35^{\prime}$&$285^{\circ}36^{\prime}$&$53^{\circ}38^{\prime}$&1.671 \\
\hline
3&$96^{\circ}31^{\prime}$&$276^{\circ}31^{\prime}$&$150^{\circ}9^{\prime}$&$330^{\circ}15^{\prime}$&$53^{\circ}41^{\prime}$&1.672 \\
\bottomrule
\end{tabular}}
\end{table}
%%%%%
\par
其中$\bar\delta=53^{\circ}40^{\prime}20^{\prime \prime}$,$\bar n=1.672$,$\sigma_{\delta}=0^{\circ}1^{\prime}20^{\prime \prime}$,再将数据代入式(19.4)中,得到$\sigma_n=0.003$.因此$n=(1.672\pm0.003)$.\par	
\textcircled{3}最后测紫光,数据如表5所示.
\newpage
%%%%%
\begin{table}[htbp]
\centering
\caption{最小偏向角法测折射率数据表——紫光}
\scalebox{1}{
\begin{tabular}{ccccccc}
\toprule
i&$\theta_9^{\prime}$&$\theta_9^{\prime \prime}$&$\theta_{10}^{\prime}$&$\theta_{10}^{\prime \prime}$&$\delta_i$&$n_i$ \\
\hline
1&$95^{\circ}30^{\prime}$&$275^{\circ}31^{\prime}$&$151^{\circ}56^{\prime}$&$332^{\circ}1^{\prime}$&$56^{\circ}28^{\prime}$&1.698 \\
\hline
2&$135^{\circ}20^{\prime}$&$315^{\circ}25^{\prime}$&$191^{\circ}50^{\prime}$&$11^{\circ}55^{\prime}$&$56^{\circ}30^{\prime}$&1.698 \\
\hline
3&$171^{\circ}35^{\prime}$&$351^{\circ}39^{\prime}$&$228^{\circ}3^{\prime}$&$48^{\circ}6^{\prime}$&$56^{\circ}27^{\prime}30^{\prime \prime}$&1.698 \\
\bottomrule
\end{tabular}}
\end{table}
%%%%%
\par
其中$\bar\delta=56^{\circ}28^{\prime}30^{\prime \prime}$,$\bar n=1.698$,$\sigma_{\delta}=0^{\circ}0^{\prime}57^{\prime \prime}$,再将数据代入式(19.4)中,得到$\sigma_n=0.003$.因此$n=(1.698\pm0.003)$.\par	
\subsection*{【分析与讨论】}
实验中的误差主要来源于最初调节分光计时是否调得精准,以及在读游标盘时的主观性,还有在实验过程中不小心的磕碰可能也会导致实验数据的偏差.
	
	
	
	
\end{document}